Теория групп = теория неабелевых групп, теория абелевых групп $\approx$ линейная алгебра.

Например,  $G=\langle x_1, x_2, \ldots, x_n \rangle$. Тогда $G \cong$ произведению циклических групп. 

Или $G = \langle a, b \rangle$,  $G$ --- абелева, тогда  $G = \{ a^k b^l \mid k, l \in \Z\}$. 

\begin{definition}
    $V$ --- конечное множество,  $|V| = n$.  $S(V) = \{f\!: V \to V (f\text{ --- биекция})\}$.

    Если $V = 1..n \implies S(V) = S_n$.
\end{definition}
\begin{definition}
    $K$ --- поле,  $n \in \N$, тогда  $GL(n, K)$ --- обратимые матрицы порядка  $n$ $= \{A\!: V \to V \mid \small \begin{array}{l} A\text{ --- биективное отображение}\\V\text{ --- }n\text{-мерное пространство}\end{array}\}$
\end{definition}
\begin{remark}
    $S_n$ вкладывается в  $GL(n, K)$.
\end{remark}
\begin{proof}
    $\pi$ --- перестановка,  $V = \langle e_1, e_2, \ldots, e_n \rangle$. $A_\pi(e_i) = e_{\pi(i)}$. Тогда  $\pi \to A_\pi$ --- инъективный гомоморфизм.
\end{proof}
\begin{theorem}[Теорема Кэли]
    Любая конечная группа изоморфна подгруппе в $S_n$ при некотором  $n$.
\end{theorem}
\begin{proof}
    Положим $n = |G|$.  $G$ --- конечная группа, то есть  $G = \{ g_1, \ldots, g_n\}$. Тогда $g \in G\quad m_g\!: G \to G \quad m_g(g_i) = gg_i$ --- биекция.

    То есть  $g \cdot g_i = g_{\pi_g(i)}, \pi_g \in S_n$ --- перестановка.

    Теперь зададим гомоморфизм:  $i\!: \begin{array}{l} G \to S_n \\ g \to \pi_g \end{array}$.  $i$ --- инъективно:  $\pi_g = \pi_{g'} \implies g \cdot e_G = g' \cdot e_G \implies g = g'$.

    Покажем, что $i$ --- гомоморфизм: надо проверить  $\pi_{g_1g_2} = \pi_{g_1} \cdot \pi_{g_2}$.

    $\pi_{hf} = \pi_h \cdot \pi_f$. \quad $g_{\pi_h(\pi_f(i))} = h \cdot g_{\pi_f(i)} = h(fg_i) = (hf)g_i = g_{\pi_{hf}(i)}$.
\end{proof}
\begin{remark}
    Заметим, что в доказательстве теоремы Кэли мы находим не обязательно минимальное регулярное представление. Например, для $S_4$ минимальное представление равно  $S_4$, а у нас  $S_{24}$.
\end{remark}
\Subsection{Смежные классы и теорема Лагранжа}
Пусть есть группа $G$ и  $H \le G$ --- подгруппа.

\begin{definition}
    Левый смежный класс по подгруппе $H$, это  $gH = \{g\cdot h \mid h \in H\}$.
\end{definition}
\begin{definition}
    Правый смежный класс по подгруппе $H$ ---  $H \cdot g = \{ h \cdot g \mid h \in H\}$
\end{definition}
Вообще говоря $gH \neq Hg$ (если $H$ --- неабелева).
\begin{example}
    Пусть $g \in H \implies gH = Hg=H$.
\end{example}
\begin{properties}[смежных классов]
    \begin{enumerate}
        \item $g_1H = g_2H \iff g_2^{-1}g_1 \in H \iff g_1^{-1}g_2 \in H$.
        \item $\forall 2$ смежных класса не пересекаются или совпадают.
    \end{enumerate}
\end{properties}
\begin{proof}
    \begin{enumerate}
        \item $g_1H = g_2H \iff \{g_1h\} = \{g_2h\} \iff \{g_2^{-1}g_1h\} = \{h\} = H$.

            $\Leftarrow\!:$ $g_2^{-1}g_1 \in H \implies \{g_2^{-1} g_1h\} = H$.

            $\Rightarrow\!:$ $\{g_2^{-1}g_1h\} = H$, подставим $h=e \implies g_2^{-1}g_1 \in H$.

            Аналогично для  $g_1g_2^{-1}$. Получили классы эквивалентности: $g_1 \underset{H}{\sim} g_2 \iff g_1H = g_2H$.
        \item Докажем отношение эквивалентности: $g_1 \underset{H}{\sim}g_2 \iff g_1 \in g_2H$.

            $\Rightarrow$.  $g_1H=g_2H$. Подставим $h=e$.

            $\Leftarrow$.  $g_1 = g_2 \cdot h_0, h_0 \in H$. $\{ g_1h \mid h \in H\} = \{(g_2h_0)h\mid h\in H\} = \{g_2(h_0h) \mid h\in H\} = \{g_2 \widetilde{h} \mid \widetilde{h} \in H\} = g_2H$.
    \end{enumerate}
\end{proof}
\begin{remark}
    Схожие утверждения верны и для правых классов.
\end{remark}
Итого: $\forall$ подгруппа  $H$ задает 2 разбиения  $G$ на смежные классы.
\begin{example}
    $G=(\Z, +), H = \langle a \rangle = \{ ka \mid k \in \Z\}$

    $l = \{l + ka \mid k \in \Z\} = \overline{l_a}$ ---  $a$ классов вычетов.
\end{example}
\begin{definition}
    $G$ --- группа,  $H$ --- подгруппа, такая что  $\exists k$ левых смежных классов,  $k$ называется индексом  $H$ в  $G$. Обозначение: $|G : H| = k$
\end{definition}
\begin{exerc}
    $gH \longleftrightarrow Hg^{-1}$ --- биекция между левыми и правыми смежными классами.
\end{exerc}
\begin{theorem}[Теорема Лагранжа]
    $|G| = n, |H| = k$,  $H \le G$.

    Тогда $|G:H| = \frac{n}{k}$. В частности $|G| \divby |H|$. Порядок группы делится на порядок подгруппы.
\end{theorem}
\begin{proof}
    Вспомним доказательство частного случая с первого модуля (порядок группы делится на порядок элемента). Мы фиксировали $a$ и разбивали все элементы на циклы длины $\ord_G a$ вида $x \to ax \to a^2x \to \ldots a^{\ord_G(a) - 1}x \to x$. В новых терминах получившиеся циклы --- классы смежности $a$ по подгруппе $H$. $\forall g \in G\!: |gH| = |H| = \ord a = k \Rightarrow |G : H| = \frac{n}{k}$
\end{proof}
\begin{definition}
    $G / H$ --- множество левых смежных классов.

    $H \setminus G$ --- множество правых смежных классов.
\end{definition}
\Subsection{Группа перестановок}
\begin{definition}
    $\pi \in S_n$ называется циклом, если  $\exists i_1, \ldots, i_k \in \{1..n\}$, такое что $\pi(i_l) = i_{l+1}$и  $\pi(i_k) = i_1$ и  $\pi(j) = j$ для остальных.
\end{definition}
\begin{definition}
    Циклы называются независимыми, если их множества подвижных точек не пересекаются.
\end{definition}
\begin{remark}
    $\pi_1, \pi_2, \ldots, \pi_n$ --- попарно независимые циклы $\implies$ их произведение не зависит от порядка множителей.
\end{remark}
\begin{theorem}
    $\pi \in S_n$.  $\pi$ --- единственным образом (с точностью до порядка) представима как произведение независимых циклов.
\end{theorem}
\begin{proof}
    Будем доказывать для $S_n \cong S(M)$.

     \begin{itemize}
         \item База, $n = 1$.
         \item Переход: $1,\ldots, n-1 \to n$.

             Рассмотрим $\pi \in S(M), a \in M$. Рассмотрим  $\pi(a), \pi(\pi(a)), \ldots$. Рассмотрим минимальное $k$, такое что  $\pi^k(a) = \pi^l(a)$ для какого-то  $0 \le l < k$./

             Если $l \neq 0$, так как  $\pi$ --- биекция, то  $\pi^{k-1}(a) = \pi^{l-1}(a)$. Противоречие.

             Если  $l = 0$, то получили цикл.  $N = \{ \pi^{i}(a) \mid i < k \}$. Пусть  $\pi_0(x) = \pi(x)$, если  $x \in N$ или  $x$ иначе. По индукционному предположению существуют циклы.
    \end{itemize}

    Единственность: ему лень :(
\end{proof}

\begin{definition}
    $\pi$ --- цикл на  $i_1, i_2, \ldots, i_k$. $\pi = (i_1 i_2..i_k)$

    Тогда $\pi \in S_n$ --- произвольная перестановка,  $\pi = (i_1 \ldots i_k)(j_1 \ldots j_l) (s_1 \ldots s_m)$
\end{definition}

\begin{theorem}
    $(ij)$ --- транспозиция. $\langle (ij) \rangle = S_n$.
\end{theorem}
\begin{proof}
    Очев $+$ доказывали
\end{proof}
\begin{theorem}
    $\langle (ijk) \rangle = A_n$ --- группа четных перестановок.
\end{theorem}
\begin{proof}
    $(ijk) = (ij)(jk) \implies (ijk) \in A_n \implies \langle (ijk) \rangle \le A_n$.

    Обратно: пусть $\pi \in A_n$.  $\pi = t_1t_2\ldots t_{2k}$, $t_i$ --- транспозиции. Достаточно доказать, что  $\forall $ транспозиций  $t_i, t_j$  $t_i \cdot t_j \in \langle (ijk)\rangle$.

    Пусть  $t_i = (ab), t_j = (cd)$. Тогда рассмотрим 3 случая:
     \begin{enumerate}
         \item $t_i = t_j \implies t_i \circ t_j = t_i^2 = \mathrm{id} \in \langle (ijk) \rangle$.
         \item  $b = c$,  $a \neq d$. Очев.
         \item $a, b, c, d$ различны. Легко показать, что  $(ab)(cd) = (cad)(abc) \in \langle (ijk) \rangle$.
    \end{enumerate}
\end{proof}

\begin{definition}
    $a \underset{H}{\equiv} b \Leftarrow a \in bH$ (далее $\equiv$ вместо $\sim$).
\end{definition}
\begin{example}
   $H = \{ n\Z \mid x\in \Z\}$ --- подгруппа.

   $b \cdot H = b + H$ --- класс вычетов по модулю  $n$. $a \in bH$  $a = b\cdot h, h \in H \iff b^{-1}a \in H$.

   $a \underset{H}{\equiv}b$, если  $a \in Hb$ ($ab^{-1} \in H$).
\end{example}
\Subsection{Факторгруппа}
\begin{definition}
    $H \le G$ называется нормальной ($H \trianglelefteq G$), если выполнено любое из трех равносильных утверждений:
    \begin{enumerate}
        \item $a \underset{H}{\equiv} b, c \underset{H}{\equiv} d \implies ac \underset{H}{\equiv} bd \quad \forall a,b,c,d \in G$.
        \item $\forall a \in G\!: aH=Ha$.
        \item  $\forall h \in H, g \in G\!: g^{-1}hg \in H$.
    \end{enumerate}
\end{definition}
\begin{definition}
    $g^{-1}hg$ и  $h$ называются сопряженными посредством  $h$.
\end{definition}
\begin{proof}
    Направления доказательства $3 \implies 1$.

     $a \equiv b \implies a = bh_1, c \equiv d \implies c = dh_2$, где $h_1, h_2 \in H$ $\implies ac = bh_1dh_2 = bdd^{-1}h_1dh_2 = bd \cdot h_3 \cdot h_2$, $h_3, h_2 \in H \implies \in bdH$.
\end{proof}
\begin{remark}
    Сопряженность --- отношение эквивалентности.
    $G = \bigsqcup\limits_i C_i$,  $C_i$ --- класс сопряженности.
\end{remark}
\begin{definition}
    $H \trianglelefteq G$. Факторгруппой  $G / H$ называется множество классов смежности со следующей операцией $(a\cdot H)(b\cdot H) = ab \cdot H$.

    Обозначение: $\overline{a_H}$,  $\overline{a_H} \cdot \overline{b_H} \coloneqq \overline{ab}_H$. Заметим, что первое определение нормальности показывает корректность данной операции.
\end{definition}
\begin{example}
    $G = S_n$.  $H \trianglelefteq S_n \implies \left[ \begin{array}{l} H = \{e\} \\ H = S_n \\ H = A_n \end{array} \right.$

     $G / G = \{ \overline{e} \}$,  $G / \{e\} \cong G$.

     $S_n / A_n$:  $e \cdot A_n = A_n$,  $(12) \cdot A_n = S_n \setminus A_n$.
\end{example}

Напоминание: ($K$ --- поле) $GL_n(K)$ --- обратимые матрицы размера  $n$,  $SL_n(K) = \{ A \in M_n(K) \mid \det A = 1\}$.

\begin{statement}
    $SL_n(K) \trianglelefteq Gl_n(K)$.

     $GL_n(K) / SL_n(K)$.  $A \cdot SL_n(K) = \{B \mid \det B = \det A \}$.

     Поэтому  $GL_n(K) / SL_n(K) \cong K^*$.
\end{statement}

\begin{statement}
    $T$ --- множество диагональных матриц, причем  $a^n = 1$. Тогда  $T \trianglelefteq SL_n(K), SL_n(K) / T = PSL_n(K)$ --- projective special group.
\end{statement}
\begin{example}
    $f = \{ \begin{array}{l} ax + b \\ cx + d \end{array} \mid ad - bc \neq 0 \}$ --- группа дробно-линейных преобразований над $K$. $f \leadsto = \begin{pmatrix} a & b \\ c & d \end{pmatrix} = A$,  $f_1 \leadsto \begin{pmatrix} a_1 & b_1 \\ c_1 & d_1 \end{pmatrix} = A_1$, $f \circ f_1 \leadsto A \cdot A_1$. То есть группа дробно-линейных преобразований ---   $PGL_2(n)$.
\end{example}
\Subsection{Теорема о гомоморфизме}
Пусть $G_1, G_2$ --- группы. $f\!: G_1 \to G_2$ --- гомоморфизм, если $f(g_1g_2) = f(g_1)f(g_2)$. Изоморфизм $\iff$ гомоморфизм + биекция.  $f$ --- автоморфизм  $\iff$ изоморфизм и  $G_1 = G_2$.
\begin{remark}
    $G$ --- абелева,  $f(g) = g^{-1}$ --- автоморфизм.

     $g_0 \in G$ --- фиксированно, $f_2(g) = g_0^{-1}gg_0$ --- автоморфизм сопряжения.
\end{remark}

$\ker f = \{g \in G \mid f(g) = e_{G_2}\}$ --- ядро гомоморфизма.  $\Im f = \{ f(g) \mid g \in G \}$ --- образ гомоморфизма.

\begin{lemma}
    $f\!: G_1 \to G_2$, $\Im f \le G_2, \ker f \trianglelefteq G_1$.
\end{lemma}
\begin{proof}
    $f(g_1) = e, f(g_2) = e$. $f(g_1g_2) = f(g_1)f(g_2) = e$.
    $f(e_{G_1}) = e_{G_2}, f(e) = f(e \cdot e) = f(e) \cdot f(e)$.

    $h \in \ker f \Rightarrow f(g^{-1}hg) = f(g^{-1}) f(h)f(g) = f(g^{-1}) f(g) = e$.

    P.S. Тут не хватает ещё проверок, но они тривиальны
\end{proof}
\begin{theorem}
    $f\!: G_1 \to G_2 \implies G_1 / \ker f \cong \Im f$.
\end{theorem}
\begin{proof}
    Возьмем $a \in \Im f$. Рассмотрим  $f^{-1}(a) = \{b \mid f(b) = a\}$. Возьмем $b_0 \in f^{-1}(a)$, тогда  $b \in f^{-1}(a) \iff f(b) = f(b_0) \iff f(bb_0^{-1}) = e \iff bb_0^{-1} \in \ker f \iff b \in b_0 \cdot \ker f$.

    $f^{-1}(a) = b_0 \ker f$.

    Построим  $\widetilde{f}\!: G_1 / \ker f  \to \Im f$. $\widetilde{f}(b \ker f) = f(b)$.
    \begin{enumerate}
         \item $\widetilde{f}$ корректно.  $c \in b \ker f \implies f(c) = f(bh) = f(b) f(h) = f(b)$.
         \item $\widetilde{f}$ --- гомоморфизм:  $\widetilde{f}(\overline{a} \cdot \overline{b}) = \widetilde{f}(\overline{ab}) = f(ab) = f(a)f(b) = \widetilde{f}(\overline{a}) \widetilde{f}(\overline{b})$.
         \item  $\widetilde{f}$ --- сюръективно.  $a \in \Im f \implies a = f(b) \implies a = \widetilde{f}(\overline{b})$.
         \item  $\widetilde{f}$ --- инъективно.  $\widetilde{f}(\overline{b}) = e \iff f(b) = e \iff b \in \ker f \iff \overline{b} = \overline{e}$.
    \end{enumerate}
\end{proof}
\begin{example}
    $G = \R^*, H = \R_+^*$.  $G / H \cong \{1, -1\} \cong \Z / 2\Z$.
     $f\!: G \to G$,  $f(x) = \frac{x}{|x|} = \text{sgn}(x)$.
\end{example}
\begin{example}
    $G = D_4$ --- группа самосовмещений квадрата. $|D_4| = 8$. Есть $1$, 3 поворота и 4 оси симетрии.

     $|G / H| = 4, G / G = F_2^2$. Первый бит --- поворот на $\frac{\pi}{2}$ и зеркаливание.
\end{example}
\begin{example}
    $G_1 / G_2$ --- группы. $G = G_1 \times G_2$, $\widetilde{G_1} = \{ (g_1, 3) \mid g_1 \in G_1 \} \cong G_1$.

    $G / G_1 \cong G_2$. $f\!: G_1 \times G_2 \to G_2$, $(g_1, g_2) \to g_2$, $\ker f = \widetilde{G_1}, \Im f = G_2 \implies G / G_1 = G_2$.
\end{example}

Пусть $G$ --- большая группа. Возьмем  $H \trianglelefteq G$, заменим  $G$ на  $(H, G / H)$. Например, может оказаться, что  $G \cong H \times G / H$.

Пример, $H \cong \Z / 2 \Z, G / H \cong \Z / 2 \Z$, то тогда либо  $G \cong \Z / 2 \Z \times \Z / 2 \Z$ и  $G \cong \Z / 4 \Z$.
\begin{definition}
    $G$ --- называется простой, если у нее нет нетривиальных нормальных подгрупп.
\end{definition}
\begin{theorem}
    $A_n$ --- проста ($n \ge 5$).
\end{theorem}
\begin{theorem}
    $PSL_n(K)$ проста для большинства  $n$.
\end{theorem}
\begin{theorem}
    \slashn
    $G$ --- конечная простая  $\implies \left[\begin{array}{ll} G \cong \Z / p \Z & p\text{ --- простое} \\ G \cong A_n & n \ge 5 \\ G \cong PSL_n(K) & K\text{ --- конечное поле} \\ \text{еще несколько матричных групп}&\\\text{еще 26 исключительных простых групп}&\end{array}\right.$
\end{theorem}
\Subsection{Действие групп}
\begin{definition}
    $G$ --- группа,  $M$ --- множество.

    Действие  $G$ на $M$ --- отображение из  $G \times M \to M$.  $(g, m) \to g \cdot m$, такая что
     \begin{enumerate}
         \item $(g_1g_2) m = g_1(g_2m)$ и $em = m$.
    \end{enumerate}
\end{definition}
\begin{definition}[Альтернативное определение]
    $G$ действует на  $M$, если задан гомоморфизм  $f\!: G \to S(M)$.
\end{definition}
\begin{remark}
    Построим $f_g\!: M \to M, m \mapsto g \cdot m$. Это биекция.
\end{remark}

Говорят $g$ действует на  $M$.  $M$ ---  $G$-множество.  $G \acts M$.
\begin{example}
    \begin{enumerate}
        \item $I_n = \{1..n\}$,  $G = S_n$.  $G \acts I_n$.
        \item  $I_n \times I_n \quad S_n \acts I_n \times I_n \quad \pi(x, y) = (\pi(x), \pi(y))$
        \item  $M=2^{I_n}$.  $S_n \acts 2^{I_n}$.  $\pi \cdot \{x_1, \ldots, x_n\} = \{\pi(x_1),\ldots, \pi(x_n)\}$.
        \item $K$ --- поле.  $K^*$ действует на  $K$ гомоморфизмами.

            ВАЖНО!!!!!!!  $a\cdot b = ab$.
        \item  $(\langle g \rangle)C_n \acts \CC$  $g \cdot z = e^{\frac{2\pi i}{n}} \cdot z$ --- поворот на $\frac{2\pi}{n}$. $g^{k}z = e^{\frac{2\pi i k}{n}z}.$
        \item $S_3 \acts \CC$.  $(123)$ --- умножение на  $-\frac{1}{2} + \frac{\sqrt{3}}{2}i$. $(12) \cdot z = \overline{z}$.
    \end{enumerate}
\end{example}
\begin{definition}
    $G$ точно действует на  $M$, если  $G \to S(M)$ --- инъективно.

    Неточное действие: $G \acts M$  $g \cdot m \coloneqq m$ --- тривиальное действие.
\end{definition}

$S_3 \acts M$ изометриями (точно).  Не $\exists$ точного действия на  $S_4 \acts M$.
\begin{example}[Примеры из теории групп]
Действие сдвигами $G \acts G\!: a \cdot b = a \cdot b$.

 $H \le G$. $G \acts G / H$,  $g_1 (g_2H) = (g_1g_2)H$.
 $\forall $ действие на чтобы то не было сводится к тому, что выше.

  $G \acts G$ сопряжениями:  $a \cdot b = aba^{-1}$. Действие автоморфизмами  $g_1(b) = aba^{-1}$.
\end{example}

\Subsection{Орбиты и стабилизаторы}
\begin{definition}
    $G \acts M$.  $m_1 \sim m_2 \Leftarrow \exists g\!: gm_1 = m_2$.
\end{definition}
\begin{statement}
    $\sim$ --- отношение эквивалентности.
\end{statement}
\begin{definition}
    Класс эквивалентности $\sim$ называется орбитой действия.
\end{definition}
\begin{definition}
    Орбита элемента --- $G \cdot m = \{g \cdot m \mid g \in G\}$.
\end{definition}
 \begin{remark}
    $M$ --- дизъюнктное объединение орбит.
    $M = \bigcup\limits_{i \in I} O_i$ --- орбиты.

    тогда $O_i$ ---  $G$-множества.  $x \in O_i, g \in G \implies gx \in O_i$ --- транзитивные множества.
\end{remark}
\begin{definition}
    Множество называется транзитивным, если $\forall m_1, m_2 \in M \exists g \in G\!: gm_1 = m_2$.
\end{definition}
\begin{definition}
    Iso$(2)$ --- группа изометрий плоскости. Транизитивно: на точки/прямые, не транзитивно на отрезки.
\end{definition}
\begin{definition}
    $G \acts M$,  $m \in M$. Стабилизатор  $G_m = \{g \in G \mid gm = m\}$.
\end{definition}
\begin{example}
    $S_4 \acts 2^{I_4}$.  $m = \{1,2\}, G \cdot m = \{ \{1,2\}, \{1,3\}, \{1,4\}, \{2,3\},\{2,4\},\{3,4\}\}, G_m = \langle (34), (12) \rangle$.
\end{example}
\begin{theorem}
    \begin{enumerate}
        \item $G_m \le G$.
        \item $m \in M, ~ n = g_0m \in Gm, ~ m=g_1n$.  Тогда $\exists$ биекция $Gm \leftrightarrow G/G_m$, причем $\{g \mid gm = n\} = g_0 G_m$, $\{g \mid gn = m \} = G_mg_1$.
        \item $|Gm| \cdot |G_m| = |G|$.
    \end{enumerate}
\end{theorem}
\begin{proof}
    \begin{enumerate}
        \item Очев.
        \item $g_0G_m \subset \{g \mid gm = n\}$ --- ясно. $\supset$: пусть  $gm = n$.  $g_0^{-1}gm = g_0^{-1}n = m \implies g_0^{-1}g \in G_m \implies g \in g_0G_m$.
        \item из $2)$ биекция: элементы орбиты $M \leftrightarrow $ смежные классы  $\implies |Gm| = |G : G_m| \implies |Gm|\cdot |G_m| = |G:G_m||G_m| = |G|$.
    \end{enumerate}
\end{proof}
\begin{example}
    $Iso(2)$ --- движения плоскости.  $S(K) = \{g \in Iso(2) \mid g(k) = k\}$,  $k$ --- квадрат.

    $g \in S(k)$ --- переставляет  $A, B, C, D$. Такая перестановка однозначно задает  $g$.
    $G \cdot A = \{A, B, C, D\}, |GA| = 4$.  $|G| = |G \cdot A| \cdot |G_A| = 4 \cdot |G_A| = 4 \cdot |(G_A)_B| = 8$.

     $G_A \acts \{B,C,D\}$.  $G_A \cdot B = \{B, D\}$.
\end{example}
\Subsection{Лемма Бернсайда}
\begin{definition}
    $Fix(g) = \{m \in M \mid gm = m\}$ --- фиксатор (множество неподвижных точек).
\end{definition}
\begin{theorem}[Лемма Бернсайда]
    $G$ --- конечная,  $G \acts M$. Тогда количество орбит действия равно среднему арифметическому размера фиксатора.
     \[
    \frac{\sum_{g \in G} |Fix(G)|}{|G|} = \text{количество орбит}
    .\]
\end{theorem}
\begin{proof}
    Рассмотрим $NotMove = \{(g, m) \mid g \in G, m \in M, gm = m\}$. Тогда  $\sum\limits_{m \in M}|G_m| = |NotMove| = \sum_{g \in G}|Fix(g)|$. Тогда: \[
    \frac{\sum_{g \in G}|Fix(g)|}{G} = \frac{\sum_{m \in M}|G_m|}{|G|} = \sum_{m \in M} \frac{|G_m|}{|G|} = \sum_{m \in M} \frac{1}{|Gm|}
    .\]
    Тогда, если рассмотреть орбиту длины $k$, то она дает вклад  $\underbrace{\frac{1}{k} + \frac{1}{k} + \ldots + \frac{1}{k}}_{k} = 1 \implies$ сумма количества орбит.
\end{proof}
\begin{example}
    Подсчет числа структур с точностью до изоморфизма.

    Закрепленное ожерелье: $F\!: \Z / 12\Z \to \{B, W\}$. Таких  $2^{12}$.
     $F_1 \sim F_2 \Leftarrow F_1(x) = F_2(x + x_0)$ при некотором $x_0$.

     Пусть  $O$ --- множество закрепленных ожерелий.  $\Z / 12\Z \acts O$. Каково число орбит?  Оно равно  $\frac{|Fix(0)| + |Fix(1)| + \ldots}{12} = $.

     $|Fix(0)| = 2^{12}$,  $|Fix(1)| = |Fix(11)| = |Fix(5)| = Fix(7) = 2^1$,  $|Fix(2)| = |Fix(10)| = 2^2, |Fix(3)| = |Fix(9)| = 2^3, |Fix(4)| = 2^4 = |Fix(8)|, |Fix(6)| = 2^6$.
\end{example}
\Subsection{Применения теории конечных групп действий}
$|G| = n$. Верно ли, что в  $G$ есть элемент порядка  $d$, $n \divby d$? Нет, иначе  бы в группе всегда был элемент порядка $n$, а значит любая группа была бы циклической, что неверно. Но если $d = p$, то ок.

Верно ли, что в  $G$ есть подгруппа порядка  $d$? Нет, например, в  $A_4$ нет подгруппы порядка $6$. Но если $d = p^k$, то ок.

\begin{theorem}[Теорема Коши]
    $|G| \divby p, p$ --- простое $\Rightarrow \exists g \in G\!: \ord g = p$
\end{theorem}
\begin{proof}
    Рассмотрим $M = \{(g_1, \ldots, g_p) \mid g_i \in G, g_1g_2\ldots g_p = e \}$. $|M| = |G|^{p-1} \divby p$, поскольку мы можем выбрать первые $p-1$ элементов произвольным образом а последним взять обратный к произведению. $C_p$ --- циклическая группа порядка $p$, $C_p \acts M\!: t(g_1, g_2, \ldots, g_p) = (g_2, \ldots, g_p, g_1) \in M, \quad t^k(g_1, g_2, \ldots, g_p) = (g_{k+1}, g_{k+2}, \ldots, g_p, g_1, g_2, \ldots, g_k) \in M$.

   Пусть $x \in M$. $|C_p \cdot x| = \frac{|C_p|}{|C_{p_x}|} = \left[ \begin{array}{l} 1 \\ p \end{array} \right.$ ($|C_p| = p$ --- простое). $|M| = \sum \text{длина орбиты} = 1+1+\ldots+1 + p + \ldots + p = a \cdot 1 + b \cdot p \divby p \Rightarrow a \divby p$. Длина орбиты равна $1$ только у элементов из $\{(g, g, \ldots, g) \mid g^p = e\}, \quad \{g \mid g^p = e\} = \{e\} \cup \{g \mid \ord g = p\}$. Поскольку длина орбиты $(e, e, \ldots, e)$ равна $1$, то $a \neq 0 \Rightarrow a \geq p$. Т.е. существует ненулевое делящееся на $p$ количество решений уравнения $x^p = e$, а значит существуют элементы порядка $p$
\end{proof}
\begin{theorem}[Первая теорема Силова]
    Пусть $|G| = p^n \cdot d, d \centernot\divby p$.

    Тогда  $\exists H \le G\!: |H| = p^n$.
\end{theorem}
\begin{proof}
    $M = \{x \subset G \mid |x| = p^n \}$.  $G \acts M$ (сдвигами). $|M| = \binom{p^nd}{p^n} = \frac{(p^nd)!}{(p^n)!(p^n(d-1))!} \centernot \divby p \implies$ длина хотя бы одной орбиты не делится на $p \implies \exists O\!: |O| \centernot \divby p$. $O = Gx$,  $|O| = \frac{|G|}{|G_x|} = \frac{p^nd}{|G_x|} \centernot \divby p \implies |G_x| \divby p^n$, но $|G_x| \le^{(*)}p^n \implies |G_x| = p^n$.

    (*) $x = \{a_1, \ldots, a_{p^n}\}$. $g \in G_x \Rightarrow g a_1 = a_i$. Выбор $i$ однозначно определяет $g = a_i a_1^{-1} \implies$ всего $\le p^n$ вариантов.
\end{proof}

{\Large \textsc{ТУТ НАЧИНАЕТСЯ ЧЕТВЁРТЫЙ МОДУЛЬ}}

Мы помним, что $G \acts G$ сдвигами, а следовательно  $G \mapsto S_n$,  $n = |G|$. 

А еще мы помним, что  можно взять $H \trianglelefteq G$ и из этого можно сделать дерево (разбивая так все группы, не являющиеся простыми) из $H, G \setminus H$. Листьями этого дерева будут простые группы. 

\begin{theorem}[Жордана-Гёльдера]
    Пусть имеется набор $H_1, \ldots, H_n$ --- простые подгруппы (подфакторы) $G$, тогда он зависит (с точностью до изоморфизма $H_i$-ых) только от исходной группы $G$. То есть не имеет значения как мы раскладываем на простые группы.
\end{theorem}
\begin{proof}
	Дана без доказательства
\end{proof}
\begin{definition}
    Разрешимая группа --- конечная группа, все простые подфакторы которой --- $\Z / p_i \Z$
\end{definition}
Это простейший случай, и в таких группах доказательства чаще всего являются индукцией по количеству факторов(покоординатная индукция), а вычисления можно делать в них покоординатно. 

Следующие по сложности варианты подфакторов --- $A_n$ --- чётные перестановки, но ладно. 
\begin{definition}
    Группа перестановок --- подгруппа $S_n$.
     $\pi_1, \pi_2, \ldots, \pi_k \in S_n$. $G = \langle \pi_i \rangle$.
\end{definition}

Мы хотим:
\begin{enumerate}
    \item $|G| = ?$ (узнать порядок порождённой группы, сколько перестановок получается из исходного набора).
    \item membership test: если  $\pi \in S_n$, то верно ли, что  $\pi \in G$. Если да, то какое разложение по базису? То есть $\pi = \prod {\pi_{i_k}}^{\pm 1}$ (элементы в разложении могут повторяться).
\end{enumerate}

Идея: 
\begin{enumerate}
    \item $G \acts \{1, 2, \ldots, n\}$. Тогда (т.к. мы знаем, что порядок группы есть мощность орбиты элемента на порядок стабилизатора этого элемента, и затем так можно и стабилизатор разбивать далее): $|G| = |G \cdot 1| \cdot |G_1| = |G\cdot 1| \cdot |(G_1)\cdot 2| \cdot |G_{1,2}| \cdot \ldots = \prod |G_{1,2,\ldots,k} \cdot (k+1)|$

        $|G\cdot 1|$ можно посчитать, нарисовав графы перестановок всех $\pi_i$ и посмотреть компоненту связности единицы.

	Но как с $|(G_1) \cdot 2|$ быть? Нужно знать системы образующих для $(G_1), (G_{12}), (G_{123})$ и прочих, но окуда брать эти системы?
\end{enumerate}
\begin{lemma}[Лемма Шрайера]
    Пусть $G$ --- конечная группа,  $H \le G$, причем $G = \langle g_1, \ldots, g_n \rangle$

    $G / H = \{x_1H, x_2H,\ldots, x_kH\}$ \text{ смежные классы, а }$x_1, x_2,\ldots, x_k$ --- система представителей смежных классов, при этом $x_1 = e$.

    Положим $\overline{g} = x_i$, если  $gH = x_iH$, т.е. для каждого элемента $g$, элемент $\overline{g}$ --- представитель класса смежности $g$. 

    Тогда $H = {\color{blue}\langle \left(\overline{g_lx_i}\right)^{-1} g_l x_i \rangle}_{\substack{i = 1..k\\ l=1..n}}$
\end{lemma}
\begin{proof}
    Нужно доказать, что сами перечисленные элементы лежат в $H$, и что любой элемент из $H$ представим в указанном виде.

    Пусть $a \coloneqq g_lx_i, b \coloneqq \overline{g_lx_i}$.  $b^{-1}a \in H$, так как  $bH=aH$, окей, значит перечисленные элементы действительно лежат в $H$.

    Осталось доказать, что любой элемент выражается через эти комбинации.

    Пусть  $h \in H$. ${\color{red}h} = g_{i_1} \cdot g_{i_2} ... g_{i_s}$

    Определим $x_{i_s} = x_1 = e$. Сначала запишем ржаку:

    {
\large
$$
x_{i_0}^{-1} {\color{red}h} = x_{i_0}^{-1} {\color{red}g_{i_1}} x_{i_1} \cdots {\color{red}g_{i_{s-2}}} x_{i_{s-2}} \cdot x_{i_{s-2}}^{-1} {\color{red} g_{i_{s-1}}} x_{i_{s-1}} \cdot x_{i_{s-1}}^{-1} {\color{red}g_{i_{s}}} x_{i_s}
$$
    }

    Что это? Запишем для каждого $l=1..s \quad x_{i_{(l-1)}}^{-1}g_{i_l}x_{i_l}$ ($x_{i_0}, \ldots, x_{i_{s-1}}$ произвольные). Заметим, что  ${\color{red}x_{i_0}^{-1}h} = x_{i_0}^{-1}g_{i_1} x_{i_1} \ldots x_{i_{(s - 1)}}^{-1} g_{i_s} x_{i_s}$ (Это исходное представление $h$ через $g_i$, поскольку $x_{i_l}$ посередине посокращались, $x_{i_s} = e$ по определению, осталось только $x_{i_0}^{-1}$ слева).

    Тогда по очереди для $l=(s\!-\!1)..0$ определим $x_{i_l} = \overline{g_{i_{l+1}} x_{i_{l+1}}} \Rightarrow$ {\color{blue}$x_{i_{(l-1)}}^{-1}g_{i_{l}} x_{i_{l}}$}~--- элемент нашей системы образующих.
    $\Rightarrow x_{i_0}^{-1}h = $ произведение наших образующих $\in H$.
    $h \in H \Rightarrow {\color{red}x_{i_0} \in H} \Rightarrow x_{i_0} = e \Rightarrow h = $ произведение наших образующих. Ура, доказали!
\end{proof}

я хочу к маме :( я тоже :( Если разобраться, то всё довольно несложно!

А теперь применим полученную лемму к нашему частному случаю.

\begin{exerc}[Частный случай (не упражнение)]
    $G \acts M$,  $m \in M$.  $G = \langle g_1, \ldots, g_n\rangle$.

    $G \cdot m = \{ m_1=m,  m_2, \ldots, m_k\}$, $\forall i \exists x_i\!: x_im = m_i$.

     $H = G_m$, тогда  $G / H = \{x_1H, x_2H, \ldots, x_k H\}$. Тогда по лемме Шрайера можно найти систему образующих для $G_m$ (стабилизатора).

     Построим \textit{сильную базу} для $G$ (последовательно набирая образующие стабилизаторов):  $G \cdot 1 = \{i_{11}, i_{12}, \ldots, i_{1k_1}\}$, $x_{1l} \cdot 1 = i_{1l}$.

     Нашли по Шрайеру образующие для  $G_1$. $G_1 \cdot 2 = \{ i_{21}, i_{22}, \ldots, i_{2k_2}\}, x_{2l} \cdot 2 = i_{2l}$ и так далее. $\{x_{ij}\}_{\substack{i=1..n-1\\j=1..k_i}}$ --- \textit{сильная база} для  $G$.
\end{exerc}

     Теперь ответим на два главных вопроса:
     \begin{enumerate}
         \item $|G| = |G_1| \cdot |G_1 \cdot 2| \cdot |G_{12} \cdot 3| \cdot \ldots = k_1 \cdot k_2 \cdots  k_{n-1} \le n! \quad (k_1 \le n, k_2 \le n-1, \ldots)$ (конкретно тут $n$ --- число слоёв в нашей сильной базе).

         \item membership test: $g \in S_n$

		 $g(1) = a_1 = x_{1a_1}(1) \text{( $x_{1a_1}$ --- элемент, который переводит 1 в $a_1$)} , x_{1a_1} \in G$. $x_{1a_1}^{-1} g \in (S_n)_1$.

        $x_{1a_1}^{-1}g(2) = a_2 = x_{2a_2}(2), x_{2a_2} \in G_1, x_{2a_2}^{-1}x_{1a_1}^{-1}g \in (S_n)_{12}$. И так далее\dots

             В конце, если $x_{n-1 a_{n-1}}^{-1} x_{n-2 a_{n-2}}^{-1} \ldots x_{1 a_1}^{-1} g= id$.

             Тогда $g = x_{1 a_1}x_{2 a_2}...x_{(n-1) a_{n-1}}$.

             Если на каком-то шаге $k$ не нашлось такого $x_{ka_k}$, то membership test провален

	     Если же дошли до конца, то получили $x_{n-1 a_{n-1}}^{-1} x_{n-2 a_{n-2}}^{-1} \cdots x_{1a_1}^{-1} g = 1$, а значит 

	     $$
	     g = x_{1a_1} x_{2a_2} \cdots x_{n-1 a_{n-1}}
	     $$
     \end{enumerate}

\begin{definition}
	Построение сильной базы и то, как мы проверяем membership test есть алгоритм Шрайера---Симса
\end{definition}
\begin{remark}
	Размер сильной базы --- не более чем квадратичен от $n$ $(k_1 + \ldots + k_{n-1} \le n + (n-1) + \ldots)$.
\end{remark}

Вопрос: $G \le S_n$. Вопрос такой: при каких $k\!:$ $\forall G \le S_n \exists$ система образующих размера $\le k$? $k = k(n)$.
\begin{exerc}
    $S_n = \langle (12), (123..n) \rangle$
\end{exerc}
\begin{example}
	$n = 2m$.  $G = \langle (12), (34), \ldots, (2m-1\ 2m)\rangle \cong (\Z / 2 \Z)^m$.

	$|G| = 2^m$,  $\forall g \in G\setminus\{e\} \ord g = 2, \quad g_1, \ldots, g_k \in G$. $|\langle g_1, \ldots, g_k \rangle| = 2^k \implies \forall$ с.о. (система образующих) имеет $\ge m = \frac{n}{2}$ элементов.

	Так мы показали, что $k(n) \ge \frac{n}2$ (если возвращаться к исходному вопросу.) Это, вероятно, и есть доказательство точности оценки.
\end{example}

\begin{theorem}
    $G \le S_n \implies G$ имеет систему образующих из $\le \left[ \frac{n}{2} \right]$ элементов.
\end{theorem}
\begin{proof}
    Я запрещаю вам доказывать данную теорему.

    Ну т.е. теорема доказывается как-то сложно, как-то с перебором и вообще со ссылкой на теорему о классификации конечных простых групп. Так что дана без доказательства.
\end{proof}
\begin{theorem}
    $G \le S_n \implies G = \langle g_1, \ldots, g_k \rangle$, $k < n$.
\end{theorem}
\begin{proof}
    $G = \langle g_1, \ldots, g_k\rangle$. Построим ориентированный граф $\Gamma$. Вершины --- числа от 1 до $n$, а рёбра породим следующим образом:

    $g_l \leadsto$ ребро  $i \to j$, если  $g_l(i) = j$ и при этом $\forall k < i \ g_l(k) = k$. Заметим, что все рёбра ведут от меньшего к большему. Т.е. нашли первую по номеру не неподвижную точку, смотрим, куда она переходит и рисуем ребро.

    Утверждение: можно выбрать образующие так, что $\Gamma$ --- без циклов (как неориентированный граф) ($\implies$ меньше $n$ ребер).

    Докажем это утверждение. Пусть нам последовательно кидают перестановки $h_1, h_2, \ldots \in G$, а мы в online строим систему образующих для них так, чтобы в соответствующем графе не было циклов. Формально говоря, по $h_1, h_2, \cdots \in G$ $\forall k$ строим  $g_{k,1}, \ldots, g_{k, i_k}$, такой что $\langle g_{k,1}, \ldots, g_{k, i_k} \rangle  = \langle h_1, \ldots, h_k \rangle$ и граф $\Gamma$ --- ацикличен. (тут $i_k$ является, по сути, размером системы образующих на шаге $k$. Почему $i$? Хороший вопрос.)

    Индукция по $k$. База --- очев. Переход  $g_{k1}, g_{k 2}, \ldots, g_{k i_k} \leadsto$ ацикличная.

    Добавляем $h_{k+1}$ и добавляем ребро в $\Gamma \leadsto \Gamma'$. $g_{k+1, i} = g_{k, i}, g_{k+1, i_{k} + 1} = h_{k+1}$ (прим. редакции: дальше вместо $k$ должно быть $k+1$, но нам пофиг)
    \begin{enumerate}
        \item $\Gamma'$ --- ацикличный $\implies$ ок.

        \item
            Пусть получился цикл (неориентированный) на вершинах $S_1, \ldots, S_m, \quad S_{i+1} = g_{k, r_i}^{\pm 1}(S_i)$ ($\pm 1$ зависит от ориентации ребра), не умаляя общности положим $S_1 = \min S_i \implies S_2 = g_{k,r_1}(S_1)$ (т.к $S_1 < S_2$).

            Заменим $g_{k,r_1}$ на $g_{k,r_m}^{\pm1} \ldots g_{k,r_2}^{\pm1} g_{k,r_1} = \widetilde{g}_{k,r_1}$. При этой замене порождённая группа не изменилась (т.к. $g_{k,r_1}$ и $\widetilde{g}_{k,r_1}$ выражаются друг через друга и остальные $g_{k,r_i}$). Тогда $\widetilde{g}_{k,r_1}(S_1) = S_1$ по построению и $\widetilde{g}_{k,r_1}(x) = x \quad \forall x < S_1$ в силу выбора $S_1$. Избавились от цикла. Если появился новый (а мог появиться только один) --- починим его и т.д.

	    Процесс остановится, так как выполняется полуинвариант: в результате замен растёт сумма номеров минимальных подвижных точек, т.к. при замене у нас у $\widetilde{g}_{k, r1}$ номер наименьшей неподвижной точки увеличился как минимум на один.
    \end{enumerate}

	    Всё, доказательство закончилось! Мы успешно разбили циклы и получили граф без циклов! А значит мы умеем выбирать образующие так, что $\Gamma$ не содержит циклов!
\end{proof}

Я потерялся очень сильно я не знаю как жить дальше.

Больше не теряйся <3

В целом, разобраться можно, но обозначения ржачные... 
\Subsection{Центр и коммутант}

\begin{definition}
    $G$ --- группа, центр  $G$  $Z(G) = \{ a \mid ag = ga\ \forall g \in G\}$.
\end{definition}
\begin{definition}
	$a, b \in G$ коммутатор  $a$ и  $b$ ---  $[a, b] = a^{-1}b^{-1}ab = (ba)^{-1}ab = \frac{ab}{ba}$ (своего рода). Т.е. <<насколько отличаются произведения в разном порядке>>
\end{definition}
\begin{definition}
    Пусть $H_1, H_2 \le G$. Тогда коммутант $[H_1, H_2] = \langle [h_1, h_2] \mid h_1 \in H_1, h_2 \in H_2\rangle$. Важно! Это подгруппа, порождённая всеми попарными коммутаторами, а не просто множество из них. Бывают случаи, когда все попарные коммутаторы не совпадают с порождённой группой.
\end{definition}
\begin{definition}
    Коммутант $G$ --- $[G, G]$.
\end{definition}
\begin{theorem}[Свойства центра и коммутанта] 
	Набор всяких тривиальных и не очень свойств  % Без этой строчки самое первое свойство выезжает на строку с заголовком теоремы, что для enumов выглядит как-то особо сомнительно
    \begin{enumerate}
        \item[$0_1$.] $G$ --- абелева  $\iff Z(G) = G$.
        \item[$0_2$.] $G$ --- абелева $\iff [G, G] = \{e\}$.
        \item[$1_1$.] $Z(g) \trianglelefteq G$.
        \item[$1_2$.] $[G, G] \trianglelefteq G$.
        \item[$2_1$.] $Z(G)$ --- абелева группа.
        \item[$2_2$.] $G / [G, G]$ --- абелева.
        \item[3.]  $H \trianglelefteq G$,  $G / H$ --- абелева  $\implies H \ge [G, G]$ (универсальное свойство коммутанта). (Расшифровка: если $H$ нормальная подгруппа в $G$, и $G$ фактор по $H$ --- абелева, то $H$ содержит коммутант $G$ в качестве подгруппы)
    \end{enumerate}
\end{theorem}
Альтернативный способ записать третье свойство: пусть $A$ --- абелева группа, тогда любой гомоморфизм $f : G \to A$ пропускается через $G / [G, G]$. Это несколько похооже на другие универсальные свойства, которые у нас были.

\begin{proof}
	Первые два свойства очевидны. Далее:

    \begin{enumerate}
        \item[$1_1$.] Нам нужно проверить замкнутость относительно умножения, наличие единицы и наличие обратного. $e \in Z(G)$ очевидно.  $a, b \in Z(G), \forall c \in G\ \  ac = ca, bc = cb \implies (ab)c = a(bc) = a(cb) = (ac)b = (ca)b = c(ab) \implies ab \in Z(G)$.

            $a \in Z(G)$, хотим проверить, что $a^{-1} \in Z(G). a^{-1}b = (b^{-1}a)^{-1} = (ab^{-1})^{-1} = ba^{-1} \implies a^{-1} \in Z(G)$. Проверили, что $Z(G)$~--- подгруппа. Нормальность проверяется легко через сопряжения: $g^{-1}hg = g^{-1}gh = h$.
        \item[$1_2$.] $[G, G] \le G$ --- по определению, остаётся проверить нормальность. 

	    Пусть $x \in [G, G]$, разберём случаи:

	    \begin{enumerate}
		    \item $x$ --- отдельный коммутант, т.е. $x = [a, b] = a^{-1}b^{-1}ab$. Рассмотрим  $g^{-1}xg = g^{-1}a^{-1}b^{-1}abg = g^{-1}a^{-1}gg^{-1}b^{-1}gg^{-1}agg^{-1}bg \text{ (просто вставили $g g^{-1}$ и $g^{-1} g$ несколько раз) } \\= (g^{-1}ag)^{-1}(g^{-1}bg)^{-1}(g^{-1}ag)^{1}(g^{-1}bg)^{1} = [a', b']$, где $a' = g^{-1}ag, b' = g^{-1}bg$. Т.е. сопряжённый к коммутанту --- тоже коммутант.

            Заметка: обратный к коммутанту --- коммутант: $[a, b]^{-1} = (a^{-1}b^{-1}ab)^{-1} = b^{-1}a^{-1}ba = [b, a] \implies \forall $ элемент  $[G, G]$ --- произведение коммутаторов, на обратные можем забивать. 

    \item $x \in [G, G]$, тогда  $x = [a_1, b_1][a_2,b_2]\ldots[a_n, b_n]$, $g^{-1}xg = g^{-1}[a_1, b_1]g g^{-1}[a_2, b_2] g\ldots g^{-1}[a_n, b_n]g = \text{ (по той же схеме, что в первом пункте) } = [a_1', b_1'][a_2', b_2']\ldots[a_n', b_n'] \in [G, G]$.

	   \end{enumerate}
	   Ура, сопряжённый к любому из $[G, G]$ лежит в $[G, G]$, а значит мы нормальны.
   \item[$2_1$.] Очев по определению центра. Элементы центра это те, которые <<абелевы со всеми>>, а значит и с другими элементами центра.
   \item[$2_2$.] Возьмём из $G / [G, G]$ два класса --- $\overline{a}$ и $\overline{b}$. Заметим, что $\overline{a} \cdot \overline{b} = \overline{a}\overline{b}\underbrace{\overline{b^{-1}a^{-1}}\overline{ba}}_{\in [G, G] \implies \overline{b^{-1} a^{-1} ba} = e} = \overline{ba}$ (схлопнули $b b^{-1}$ и $a a^{-1}$).
         \item[3.] $G / H$ --- абелева  $\forall a, b \in G\quad \overline{a_H} \overline{b_H} = \overline{b_H}\overline{a_H}$.  $\overline{a_H}^{-1}\overline{b_H}^{-1} \overline{a_H}\overline{b_H} = \overline{e_H} \implies \overline{a^{-1} b^{-1}ab} = \overline{e_H} \implies a^{-1}b^{-1}ab \in H$, т.е. $[a, b] \in H \implies [G, G] \le H$.
         \item[3 Alt] Несложное упражнение на языке стрелочек.
    \end{enumerate}
\end{proof}

Тут нужна картинка, но я краб

А я --- рак, но в общем тут две картинки, где мы рисуем дерево из групп и фактор групп: Есть $G$, её раскладываем на $Z(G)$ и $G / Z(G)$ или на $[G, G]$ и $G / [G, G]$, при этом $Z(G)$ и $G/[G, G]$ --- абелевы, и их мы раскладываем дальше.

\begin{theorem}
    $A$ --- конечно абелевая  $\implies$  $A \cong \Z / p_1^{\alpha_1} \times \Z / p_2^{\alpha_2} \times \ldots \times \Z / p_k^{\alpha_k}$.
\end{theorem}
\begin{proof}
    Теорема дана без доказательства (пока что).
\end{proof}
\begin{example}
    $S_n$,  $n \ge 5$.
    $[S_n, S_n] = A_n$,  $S_n / A_n = C_2$,  $[A_n, A_n] = A_n$ --- соверешенная группа.

     $Z(S_n) = \{e\} = Z(A_n) = \{e\}$.

     $n = 4$.  $[S_4, S_4] = A_4$.  $[A_4, A_4] = D_4 = \langle (12), (34), (13), (24) \rangle$. $[D_4, D_4] = \{e\}$.
\end{example}
\begin{example}[Некоторые цепочки групп]
	Рассмотрим несколько цепочек групп, получаемых из конечной группы $G$:

	\begin{enumerate}
		\item $G_1 =G / Z(G) \to G_2 = G_1 / Z(G_1) \to \ldots G_k = G_{k-1} / Z(G_{k-1})$.

    \item $G \to G_1 = [G, G] \to G_2 = [G, G_1] \to G_i = [G, G_{i-1}]$.

    \item $G_{k+1} = [G_k, G_k]$.
\end{enumerate}
\end{example}
\begin{statement}

	О том, куда могут приводить цепочки:
    \begin{enumerate}
        \item Первая цепочка приходит к $\{e\}$ $\iff$ вторая цепочка приходит к $\{e\}$.
        \item Вторая цепочка приходит к $\{e\} \implies$ третья цепочка приходит к  $\{e\}$.
    \end{enumerate}
\end{statement}
\begin{definition}
	Если существует $k$, такое  $G_k = \{e\}$(первая или вторая цепочки привели к $\{e\}$), то  $G$ называется нильпотентной группой.
    Если существует $k$, такое  $G_k = \{e\}$(в третьей цепочке), то  $G$ называется разрешимой группой.
\end{definition}
\begin{remark}
	Из нильпотентности следует разрешимость, но только в одну сторону.
\end{remark}
\begin{example}
    $S_4$ --- разрешима, но не нильпотентна.
\end{example}
\begin{example}
	Рассмотрим верхнетреугольные матрицы {\color{red} с единицами на диагонали}  ($UT_n(K)$) $\le GL_n(K)$  --- {\color{red} нильпотентная} подгруппа.

    А вот $GL_n(K)$ --- не нильпотентная и даже не разрешимая группа, проверим это через факторизацию по центру (первая цепочка), и через третью цепочку:

    $Z(GL_n(K)) = \{k\cdot E\}$ --- скалярные матрицы (то есть единичная + диагональные).  $GL_n / Z(GL_n) = PGL_n$,  $Z(PGL_n) = \{e\}$. ($PGL_n$ --- проективная линейная группа). А раз центр обнулился, то дальнейшая факторизация по нему уже ни к чему не приведёт. 

     $[GL_n, GL_n] = SL_n(K), [SL_n, SL_n] = SL_n(K)$.
\end{example}
\begin{theorem}
    $G$ --- конечна. Тогда

    Следующие условия равносильны:
    \begin{enumerate}
        \item $G$ --- разрешима (то есть разлагается на простые абелевы группы).
        \item Все простые подфакторы $G$ --- это  $\Z / p_i \Z$. (по теореме Жордана-Гёльдера знаем, что разложение на простые подфакторы у конечных групп будет зависеть только от самой исходной группы)
    \end{enumerate}
\end{theorem}
\begin{proof}
    $1 \implies 2$. $G$ раскладывается на $G / [G, G]$ (абелева) и $G_1 = [G, G]$, далее $G_1$ раскладывается на $G_1 / [G_1, G_1]$ и $G_2 = [G_1, G_1]$, и так далее... В свою очередь все $G_i / [G_i, G_i]$ раскладываются на подфакторы~--- $\Z / p_j \Z$.

    $2 \implies 1$. Пусть $G$ неразрешима. Тогда  $\exists i\!: [G_i, G_i] = G_i \neq \{e\}$. Все подфакторы  $G$ абелевы  $\iff$ все подфакторы  $G_i$ абелевы.  $G_i = H$.

    Разложим  $H$ на $H_1 = [H, H]$ и $H_2 = H / [H, H]$. $H_2$ разложим ещё на $H_3$ и $H_2 / H_3$, и так далее... Спускаемся по дереву, в ветке с $H_2$: если $[H, H] = H$, то $[H / U, H / U] = H / U$, то есть всегда будет $U: [U, U] = U \implies$ никогда не будет абелевой группы.
\end{proof}
\begin{theorem}
    $G$ --- конечная, тогда  $|G| = p^n \implies G$ --- нильпотентна.
\end{theorem}
\begin{proof}
    $G \to G / Z(G) = G / Z(G) = G_1 \leadsto G_1 / Z(G_1) = G_2 \leadsto \ldots$, $G_i = p^{\alpha_i} \quad \forall i$. Достаточно доказать:  $G_i \neq \{e\} G_{i+1} \neq G_i$, то есть  $Z(G_i) \neq \{e\}$.

    То есть  Th  $\Leftarrow$  лемма.
\end{proof}
\begin{lemma}
    $|G| = p^n$,  $p$ --- простое  $\implies Z(G) \neq \{e\}$.
\end{lemma}
\begin{proof}
    Пусть $G \acts G$ сопряжениями. $g \cdot m = g m g^{-1}$.   $O_i$ --- орбита,  $O_1 = G \cdot e = \{e\}$.  $|O_1| + |O_2| + \ldots = p^n \implies \exists i \neq 1\!: |O_i| = 1$, так как любая орбита имеет порядок либо $1$, либо степень $p$. Тут важно пояснить, что если $gxg^{-1} = x$, то  $gx = xg$, а если орбита  $=1$, то это выполняется для всех  $g \implies x\in Z(G)$.

    Теперь пусть $|G| = p_1^{\alpha_1} \ldots p_k^{\alpha_k}$. По первой теореме Силова $\exists |G_1| = p_1^{\alpha_1}, |G_2| = p_2^{\alpha_2}, \ldots$.
\end{proof}
\begin{theorem}
    $|G| = p^2 \implies G$ --- абелева.
\end{theorem}
\begin{proof}
    $|G| = p^2$,  $\exists g\!: \ord g =p^2 \implies G$ --- циклическая  $\cong \Z / p^2 \Z \ldots$

    Иначе $\forall g \neq e \ord g = p$.  Тогда либо $Z(G) = G$ (тогда $G$ --- абелева), либо $|Z(G)| = p$ (такого быть не может). Тогда  $Z(G) = \langle a \rangle, \ord a = p$,  $b \notin Z(G), \ord b = p$.  $\langle a, b\rangle = p^2 \implies \langle a, b \rangle = G \implies G$ --- абелева.
\end{proof}
\begin{remark}
    $G$~--- абелева $\implies G \cong \Z / p^2 \Z$ или $\Z / p \Z \times \Z / p \Z$.
\end{remark}
\begin{example}
    $|G| = p^3$.
    Пусть $H \trianglelefteq G$.  $H = \langle x, y \mid x^p = y^p = e, xy=yx\rangle \cong \Z / p\Z \times \Z / p\Z$.

     $\exists z \in G \setminus H, \ord z = p$, $\langle x, y, z\rangle = G = \{ x^k y^l z^m \}$.

     $H \trianglelefteq G \implies g^{-1}x^{k}y^{l}g = x^{k'}y^{l'}$.

     Тогда $z^{-1}hz = f(h) \in H$, знание  $f$ однозначно задает  $G$.  $z^{-1}hz = f(h), hz = zf(h)$.

     Наблюдение: $H \trianglelefteq G, g \in G$.  $f_g(h) = g^{-1}hg$ --- автоморфизм  $H$. $(f_g(h))^{-1} = f_{g^{-1}}(h)$.
\end{example}
\begin{definition}
    $Aut(G)$ --- группа автоморфизмов  $H$ (относительно композиции).
\end{definition}
\begin{statement}
    $G$ --- группа.  $G_1, G_2 \le G$. $f\!: G_1 \times G_2 \to G$ и $(g_1,g_2) \to g_1g_2$.

    $f$ --- изоморфно (тогда $\implies G \cong G_1 \times G_2$) $\iff$
    \begin{enumerate}
        \item $G_1+G_2 = G$.
        \item $G_1 \cap G_2 = \{e\}$.
        \item $g_1g_2 = g_2g_1\quad \forall g_1 \in G_1, g_2 \in G_2$.
    \end{enumerate}
\end{statement}
\begin{exerc}
    Пусть знаем $1, 2$ --- выполнены, тогда  $3 \iff G_1 \trianglelefteq G, G_2 \trianglelefteq G$.
\end{exerc}
\begin{definition}
    Если все то же, что и выше, но только одна из $G_1, G_2$ нормальна, то $G$ называется внутренним полупрямым произведением  $G_1$ и $G_2$.

    $g = g_1g_2$ и $G$~--- нормальная, $(g_1g_2) \cdot (g_1' g_2') = g_1g_1'(g_1')^{-1}g_2g_1'g_2' = (g_1g_1')((g_1')^{-1}g_2g_1')g_2' = \underbrace{(g_1g_1')}_{\in G_1}\underbrace{(f_{g_1'}(g_2)g_2')}_{\in G_2}$.

    Полупрямое произведение задаётся $G_1, G_2$ и гомоморфизмом $G_1 \to Aut(G_2)$.
\end{definition}
