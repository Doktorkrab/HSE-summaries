А теперь забыли про евклидовость и пошли в другом направлении.

\Subsection{Аффинное пространство}

Что такое трёхмерное пространство? $\R^3$ --- пространство с началом координат и базисом. Можно опускаться в координаты и взять $V$ над $\R$, что $\dim V = 3$ и ничего про базис не говорить, но нам всё равно нужно начало координат. Избавимся от всех этих условностей (и от начала координат)

\begin{definition}
	Пусть $K$ --- поле, тогда Аффинным пространством над $K$ называется тройка $(V, \vec{V}, t)$, где $V$ --- множество (точки), $\vec{V}$ --- векторное пространство над $K$ (векторы), и $t: V \times \vec{V} \to V$ --- откладывание вектора от точки.

	И при этом должны выполняться свойства:

	\begin{enumerate}
		\item[1,2.] $\vec{V}$ действует на $V$ (группа векторов действует на множество точек)
		\item[3] $\forall p_1, p_2 \in V \exists ! v \in \vec{V} : t(p_1, v) = p_2$. Такой $v$ называется $(\overrightarrow{p_1 p_2})$
	\end{enumerate}

	$\vec{V}$ называется векторной частью, $V$ --- аффинной частью, а $\dim \vec{V}$ --- размерностью пространства.
\end{definition}

\begin{example}
    $V$ --- векторное пространство, то $(V, V, t)$, где $t(x, y) = x + y$ --- аффинное пространство.
\end{example}
\begin{example}
    $V$ --- векторное пространство, $U \le V$, $v_0 \in V$, тогда $(v_0 + U, U, t)$ --- аффинное подпространство векторного пространства. $t(v_0 + x, y) = v_0 + (x + y) \in v_0 + U$
\end{example}
\begin{definition}
    Векторизация.

    Пусть $(V, \vec{V}, t)$ --- аффинное пространство, $O \in V$ --- любая точка. Тогда по третьей аксиоме $V_O(X) := \overrightarrow{OX}$ --- биекция, а по аксиомам 1-2 это изоморфизм $(V, \vec{V}, t)$ и пространства из примера 1. Связь между векторизациями: $O, O' \in V$, тогда

    $V_O'(X) = \overrightarrow{O'X} = \overrightarrow{O'O} + \overrightarrow{OX} = \overrightarrow{O'O} + V_O(X)$, т.е. две векторизации отличаются между собой сдвигом на разность начал координат.
\end{definition}

\begin{definition}
	$(v_1, v_2, \cdots, v_n) \to \sum a_i v_i$ --- линейная комбинация
\end{definition}

\begin{definition}
	Если у линейной комбинации $\sum a_i = 0$, то данная линейная комбинация называется сбалансированной. Пример: $v_1 - v_2$
\end{definition}
\begin{definition}
	Если у линейной комбинации $\sum a_i = 1$, то данная линейная комбинация называется барицентрической комбинацией. Пример: $\frac12 v_1 + \frac12 v_2$
\end{definition}

\begin{statement}
	Любая барицентрическая комбинация является точкой (не зависит от векторизации)

	Любая сбалансированная комбинация --- вектор 
\end{statement}

\begin{definition}
	Аффинное отображение

	$f: (V, \vec{V}, t_V) \to (U, \vec{U}, t_U)$ --- аффинное отображение, если $\exists$ векторизации $v \in V, u \in U, f(v) = u$ (зафиксируем начало координат так, чтобы начало координат переходило в начало координат) такие, что $\widetilde{f} : \vec{V} \to \vec{U}$ --- линейно. $\widetilde{f}: \vec{V} \to V \to U \to \vec{U}$, где за первую стрелочку отвечает $V_u^{-1}$, за вторую --- $f$, а за третью --- $V_u$, первое и третье --- векторизация (с обратной).
\end{definition}

\begin{definition}
	Пусть $(V, \vec{V}, t)$ --- афинное пространство, а $(U, \vec{U}, t')$ --- его подпространство, если $U \subset V$, $\vec{U} \leq \vec{V}$ и $t \Big|_{U \times \vec{U}} = t'$
\end{definition}

\begin{definition}
    Аффинная оболочка $p_1, \cdots, p_k \in V$ это наименьшее аффинное подпространство $(U, \vec{U}, t')$ такое, что $\forall i p_i \in U$
\end{definition}

\begin{definition}
    Прямая --- оффинная оболочка пары точек.
\end{definition}

\begin{theorem}[Мёбиуса]
   $f : V \to U$ т.ч. $\forall$ прямой $l \in V$ $f(l)$ --- тоже прямая и $f$ --- инъективна. Наше поле --- $K = \R$. Тогда $f$ --- аффинное.
\end{theorem}

\begin{remark}
    Факт: Аффинная оболочка $p_1, \cdots, p_k$ --- это множество всех их барицентрических комбинаций.
\end{remark}

\begin{example}
	Так, аффинная оболочка точек $p_1, p_2$ имеет вид $\{ tp_1 + (1-t)p_2 \forall t \in K (\text{в нашем случае --- }\R) \}$
\end{example}
